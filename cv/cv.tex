\documentclass[11pt, a4paper]{article}
%\usepackage{fontspec} 
\usepackage{etaremune}

% DOCUMENT LAYOUT
\usepackage{geometry} 
\geometry{a4paper, textwidth=6.1in, textheight=10.2in, marginparsep=7pt, marginparwidth=.6in}
\setlength\parindent{0in}

\usepackage[spanish]{babel}
\usepackage[bookmarks, colorlinks, breaklinks, 
% ---- FILL IN HERE THE TITLE AND AUTHOR
	pdftitle={Javier Acevedo - vita},
	pdfauthor={Javier Acevedo},
]{hyperref}

\usepackage[utf8]{inputenc} % Acepta caracteres en castellano
\usepackage[T1]{fontenc} % Encoding de salida al pdf

\usepackage{marginnote}
\newcommand{\years}[1]{\marginnote{\scriptsize #1}}
\renewcommand*{\raggedleftmarginnote}{}
\setlength{\marginparsep}{7pt}


% ---- CUSTOM COMMANDS

\newcommand{\html}[1]{\href{#1}{\scriptsize\textsc{[html]}}}
\newcommand{\pdf}[1]{\href{#1}{\scriptsize\textsc{[pdf]}}}
\newcommand{\doi}[1]{\href{#1}{\scriptsize\textsc{[doi]}}}

\renewcommand*{\raggedleftmarginnote}{}
\setlength{\marginparsep}{7pt}
\reversemarginpar









\begin{document}
\begin{center}{\huge \bf Javier Alejandro Acevedo Barroso}\\[1cm]\end{center}
\begin{minipage}[t]{0.465\textwidth}
  Teléfono: (+57) 301-680-9844 \\
  Email: \href{mailto:ja.acevedo12@uniandes.edu.co}{ja.acevedo12@uniandes.edu.co}\\
  Email: \href{mailto:jaalacba@hotmail.com}{jaalacba@hotmail.com      }\\
\end{minipage}

\hrule

\section*{Información Personal}
Nacido en Bucaramanga, Colombia, el 4 de enero de 1997 (21 años de edad).\\


\section*{Educación}
\noindent
\years{2015-2019}\textsc{Pregrado en Física}\\ {\emph{Institución}}: Departamento de Física, Universidad de los Andes. {\emph{Tesis}}: Simulación de materia oscura colisional con un método de Lattice-Boltzmann. {\emph{Director}}: Dr. Jaime Forero.\\

\section*{Participación en Eventos}
\years{2018}{Escuela de Astronomía Uniandes 2018.}\\ {\emph{Institución}}: Departamento de Física, Universidad de los Andes. \\
\years{2018}{MOCa 2018: Materia Oscura en Colombia}\\ {\emph{Institución}}: Departamento de Física, Universidad de los Andes. {\emph{Poster}}: Simulating Collisional Dark Matter.\\

\section*{Actividad de Investigación}
\years{2018}{Simulación de materia oscura colisional con un método de Lattice-Boltzmann.}\\ {\emph{Institución}}: Departamento de Física, Universidad de los Andes. {\emph{Director}}: Dr. Jaime Forero.\\	
\years{2017}{Caracterización de materiales utilizando tomografía de Muones}\\ {\emph{Institución}}: Departamento de Física, Universidad de los Andes. {\emph{Director}}: Dr. Carlos Ávila.\\

\section*{Experiencia Docente}
\years{2017-2018}{Tutor de la Clínica de Problemas de Física.}\\ {\emph{Institución}}: Departamento de Física, Universidad de los Andes. {\emph{Supervisor}}: Juan Diego Arango Montoya.\\	

\section*{Reconocimientos y Becas}
\years{2014}{Beca Bachilleres por Colombia, Programa Mario Galán Gómez } otorgada por Ecopetrol.\\
\years{2013}{Mejor estudiande del departamento de Santander. Prueba Saber 11 2013. Otorgado por el Ministerio de Educación.}\\ 

\section*{Habilidades Adicionales}
\begin{itemize}
\item Lenguajes: Español (nativo), Inglés (C1), Alemán (A1).
\item Sistemas operativos Linux y Windows. 
\item Lenguajes de programación: C, Python, Java y Bash.
\item Uso de software adicional: IRAF, \LaTeX.
\end{itemize}


\end{document}