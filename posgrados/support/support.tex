%%%%%%%%%%%%%%%%%%%%%%%%%%%%%%%%%%%%%%%%%
% Thin Formal Letter
% LaTeX Template
% Version 2.0 (7/2/17)
%
% This template has been downloaded from:
% http://www.LaTeXTemplates.com
%
% Original author:
% WikiBooks (http://en.wikibooks.org/wiki/LaTeX/Letters) with modifications by 
% Vel (vel@LaTeXTemplates.com)
%
% License:
% CC BY-NC-SA 3.0 (http://creativecommons.org/licenses/by-nc-sa/3.0/)
%
%%%%%%%%%%%%%%%%%%%%%%%%%%%%%%%%%%%%%%%%%

%----------------------------------------------------------------------------------------
%	DOCUMENT CONFIGURATIONS
%----------------------------------------------------------------------------------------


\documentclass[10pt]{letter} % 10pt font size default, 11pt and 12pt are also possible

\usepackage{geometry} % Required for adjusting page dimensions

%\longindentation=0pt % Un-commenting this line will push the closing "Sincerely," to the left of the page

\geometry{
	paper=letterpaper, % Change to letterpaper for US letter
	top=3cm, % Top margin
	bottom=1.5cm, % Bottom margin
	left=4.5cm, % Left margin
	right=4.5cm, % Right margin
	%showframe, % Uncomment to show how the type block is set on the page
}

\usepackage[T1]{fontenc} % Output font encoding for international characters
\usepackage[utf8]{inputenc} % Required for inputting international characters

\usepackage{stix} % Use the Stix font by default
\usepackage[spanish]{babel}

\usepackage{setspace}
\doublespacing
\usepackage{microtype} % Improve justification
\usepackage{quotmark} %Uso consistente con la RAE de comillas
%----------------------------------------------------------------------------------------
%	YOUR NAME & ADDRESS SECTION
%----------------------------------------------------------------------------------------

\signature{Javier Alejandro Acevedo Barroso} % Your name for the signature at the bottom

\address{Calle 10 \# 2 - 26\\ Bogotá, Colombia} % Your address and phone number

%----------------------------------------------------------------------------------------

\begin{document}

%----------------------------------------------------------------------------------------
%	ADDRESSEE SECTION
%----------------------------------------------------------------------------------------

\begin{letter}{\\ Comité de Posgrados en Física \\ Carrera 1 \# 18A-10, Bloque Ip \\ Bogotá, Colombia} % Name/title of the addressee

%----------------------------------------------------------------------------------------
%	LETTER CONTENT SECTION
%----------------------------------------------------------------------------------------

\opening{\textbf{Señores Comité de Posgrado,}}
Con esta carta me gustaría expresar mi interés en solicitar ayuda financiera de parte del Departamento de Física para cursar el programa de Maestría en Ciencias-Física.


Nací y me crié en Bucaramanga, ciudad en la que aún viven mis padres y la gran mayoría de mi familia cercana.  Inicié mis estudios de física en la Universidad Industrial de Santander (UIS) en 2014 pues la idea de estudiar fuera de Bucaramanga era imposible por falta de recursos. Sin embargo, ese mismo año gané la beca \tqt{Bachilleres por Colombia, programa Mario Galán Gómez} otorgada por Ecopetrol, cuyo proceso de selección está basado en el puntaje de la prueba Saber 11 y el nivel socioeconómico del candidato. La beca me permitió transladarme a Bogotá y estudiar en la Universidad de los Andes, pues cubrió la matrícula completa de la universidad, gastos de transporte, un auxilio de vivienda y un auxilio para libros.


Al solicitar admisión a la Maestría, lo hago pensando en dedicarme completamente al trabajo de  investigación, de aprendizaje, y de docencia. Mis razones para solicitar ayuda financiera son: la incapacidad de sostenerme en Bogotá sin un trabajo de tiempo completo, y que no cuento con los recursos para poder cubrir los gastos de la matrícula. En el pasado he estudiado exitosamente con financiación, mostrando que tengo la capacidad tanto cognitiva como psicológica para cumplir los objetivos del programa y los requisitos de permanencia de la beca.


\vspace{2\parskip} % Extra whitespace for aesthetics
\closing{Atentamente,}
\vspace{2\parskip} % Extra whitespace for aesthetics

%\ps{P.S. You can find additional information attached to this letter.} % Postscript text, comment this line to remove it

%----------------------------------------------------------------------------------------

\end{letter}
 
\end{document}