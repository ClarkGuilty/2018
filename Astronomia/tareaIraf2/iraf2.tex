\documentclass[12pt]{article}

\usepackage{graphicx}
\usepackage{epstopdf}


\usepackage[spanish]{babel} % silabea palabras castellanas <- Puedo poner comentarios para explicar de que va este comando en la misma línea
\selectlanguage{spanish} 

%Encoding
\usepackage[utf8]{inputenc} % Acepta caracteres en castellano
\usepackage[T1]{fontenc} % Encoding de salida al pdf

%Triunfó el mal
\usepackage[normalem]{ulem}
\useunder{\uline}{\ul}{}
\providecommand{\e}[1]{\ensuremath{\times 10^{#1}}}
\usepackage{quotmark} %Uso consistente con la RAE de comillas
\usepackage{listings} % Comandos de la terminal

\usepackage{textcomp}
\usepackage{gensymb}


%Hipertexto
\usepackage[colorlinks=true,urlcolor=blue,linkcolor=blue]{hyperref} % navega por el doc: hipertexto y links

%Aquello de las urls
\usepackage{url} 

%simbolos matemáticos
\usepackage{amsmath}
\usepackage{amsfonts}
\usepackage{amssymb}
\usepackage{physics} %Best pack

% permite insertar gráficos, imágenes y figuras, en pdf o en eps
\usepackage{graphicx}
\usepackage{epstopdf}
\usepackage{multirow}
\usepackage{float}
\usepackage[export]{adjustbox}
% geometría del documento, encabezados y pies de páginas, márgenes
\usepackage{geometry}
\usepackage{comment}

%\usepackage[english]{babel}
%\usepackage[latin5]{inputenc}
% \usepackage{hyperref}
%\newdate{date}{10}{05}{2013}
%\date{\displaydate{date}}
\begin{document}




\title{Cúmulos Abiertos \\ Reducciones CCD 1 IRAF}

\author{
\textbf{Javier Alejandro Acevedo Barroso\thanks{e-mail: \texttt{ja.acevedo12@uniandes.edu.co}}}\\
\textit{Universidad de los Andes, Bogotá, Colombia}\\
 }% Hasta aquí llega el bloque "author" (son dos autores por informe, orden alfabético)

\date{\today}
%\date{Versión $\alpha \beta$ fecha del documento}
\maketitle %Genera el título del documento


\normalsize
\newpage


\section{Reducción CCD}

El objetivo del ejercicio es realizar la reducción de imágenes tomadas con CCD (Charged-Coupled Device). Tales reducciones pueden realizarse de dos maneras utilizando IRAF: calculando cada etapa de la reducción con tareas específicas de IRAF para ello, o calcular directamente la corrección con el paquete CCDRED de IRAF.



\subsection{Primer Camino}
Para realizar el proceso paso a paso, se debe empiezar definiendo la región del overscan que se utilizará en la corrección. Usualmente, a las imágenes astronómicas se les añade una serie de columnas (y/o filas) que corresponden a la lectura del CCD con cero fotones recolectados, es decir, el nivel \tqt{cero} u \tqt{offset}\cite{handbookCCD}. En el caso del ejercicio, se toma la imágen \tqt{92006.fits} y manualmente se busca su región de overscan. Esta región corresponde a las columnas 320 hasta el final de la imágen en 352. Adicionalmente, hay un gran pico de conteos en la columna 319, justo antes de empezar la región del overscan. Esto se puede observar en la siguiente imagen. \\

\begin{figure}[H]
  \centering
   \includegraphics[scale= 0.5]{im01.png}
%  \caption{}
  \label{im01}
\end{figure}



Una vez identificad, se procede a definir los límites de la imagen real, es decir, la imagen sin el overscan y el pico que separa el overscan. Tras inspeccionar la imagen se decidió que las columnas a promediar para el \tqt{offset} son de la 325 hasta la 350, esto con el fin de cubrir la mayor cantidad de columnas posibles sin incluir el pico. Para las filas se seleccionó desde la 4 hasta la 508, en este caso simplemente se dejó 4 pixeles de tolerancia con respecto a los límites de la imagen.\\

Adicionalmente a calcular el \tqt{offset}, también se debe recortar las imagenes para eliminar la región del overscan y otras impurezas (en este caso el pico que separa el overscan). Por lo anterior, se debe decidir las nuevas fronteras de la imagen recortada. En este caso se tomó desde la línea 2 hasta la 510, y desde la columna 2 hasta la 316. A continuación se puede observar la imagen incluyendo el overscan.


\begin{figure}[H]
  \centering
   \includegraphics[scale= 0.5]{im02.png}
%  \caption{•}
  \label{im02}
\end{figure}


Una vez definidas las fronteras de los cortes y los cálculos, se procede a realizar los mismos. Para ello, se utiliza la tarea \tqt{colbias} de \tqt{noao.imred.bias} . Para usar la tarea se debe especificar el nombre de la imagen recortada, el nombre de la nueva imagen, la región a partir de la cual se calculará el \tqt{offset}, la región a preservar de la imagen, la forma en la que se interpolarán los datos, la dispersión a partir de la cual se eliminan datos, entre otros. A continuación la misma imagen anterior tras usar colbias. Nótese que la imagen ya no contiene la region del overscan y su brillo ha cambiado ligeramente, esto último debido a la corrección del \tqt{offset}.



\begin{figure}[H]
  \centering
   \includegraphics[scale= 0.5]{im03.png}
%  \caption{•}
  \label{im02}
\end{figure}



\bibliography{bibTes}{}
\bibliographystyle{unsrt}



\end{document}




\begin{figure}[H]
  \centering
   \includegraphics[scale= 0.65]{im03.png}
  \caption{Cargando las imágenes a diferentes frames de DS9 desde la sesión de IRAF. }
  \label{im03}
\end{figure}





\section{Cronograma}

\begin{table}[htb]
	\begin{tabular}{|c|cccccccccccccccc| }
	\hline
	Tareas $\backslash$ Semanas & 1 & 2 & 3 & 4 & 5 & 6 & 7 & 8 & 9 & 10 & 11 & 12 & 13 & 14 & 15 & 16  \\
	\hline
	1 & X & X & X  &   &   &   &   &  &  &   &   &   &   &   &   &   \\
	2 &   &  & X & X & X &  &  &   &   &  &  &  &   &  &  &   \\
	3 &   &   &   &  & X  & X  & X  & X &   &   &   &  &   &   &  &   \\
	4 &  &  &  &  &  &  &  & X & X & X & X &   &   &   &   &   \\
    5 &  &  &  &  &  &  & X & X &  &  &  &   &   &   &   &   \\
	6 &   &   &   &   &  &   &  X & X  &  &   &  X & X &  X & X  & X &   \\
	\hline
	\end{tabular}
\end{table}
\vspace{1mm}
 