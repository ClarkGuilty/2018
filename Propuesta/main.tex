\documentclass[12pt]{article}

\usepackage{graphicx}
\usepackage{epstopdf}
\usepackage[spanish]{babel}


\usepackage[spanish]{babel} % silabea palabras castellanas <- Puedo poner comentarios para explicar de que va este comando en la misma línea

%Encoding
\usepackage[utf8]{inputenc} % Acepta caracteres en castellano
\usepackage[T1]{fontenc} % Encoding de salida al pdf

%Triunfó el mal
\usepackage[normalem]{ulem}
\useunder{\uline}{\ul}{}
\providecommand{\e}[1]{\ensuremath{\times 10^{#1}}}

\usepackage{textcomp}
\usepackage{gensymb}


%Hipertexto
\usepackage[colorlinks=false,urlcolor=blue,linkcolor=blue]{hyperref} % navega por el doc: hipertexto y links

%Aquello de las urls
\usepackage{url} 

%simbolos matemáticos
\usepackage{amsmath}
\usepackage{amsfonts}
\usepackage{amssymb}
\usepackage{physics} %Best pack

% permite insertar gráficos, imágenes y figuras, en pdf o en eps
\usepackage{graphicx}
\usepackage{epstopdf}
\usepackage{multirow}
\usepackage{float}
\usepackage[export]{adjustbox}
% geometría del documento, encabezados y pies de páginas, márgenes
\usepackage{geometry}
\usepackage{comment}

%\usepackage[english]{babel}
%\usepackage[latin5]{inputenc}
% \usepackage{hyperref}
%\newdate{date}{10}{05}{2013}
%\date{\displaydate{date}}
\begin{document}

\begin{center}
\Huge
Simulación de materia oscura colisional con un método Lattice-Boltzmann

\vspace{3mm}
\Large Javier Alejandro Acevedo Barroso

\large
201422995


\vspace{2mm}
\Large
Director: Jaime Ernesto Forero Romero

\normalsize
\vspace{2mm}

\today
\end{center}


\normalsize
\newpage
\section{Introduccion}
La cosmología moderna infiere que la materia ordinaria (materia
bariónica) representa s\'olo el 5 \% de la energía del universo, la
energía oscura representa el 68 \% y la materia oscura el 27 \%
\cite{planckCitetion}.   
La presencia de la materia oscura se infiere a trav\'es de los efectos
gravitacionales que produce, pero a\'un no es claro a que tipo de
part\'icula fundamental por fuera del modelo est\'andar debe
corresponder. 

Sin embargo, es posible simular numéricamente la evoluci\'on temporal
de una distribuci\'on arbitraria de diferentes tipos posibles de
materia oscura. 
Por ejemplo,  simulaciones numéricas permitieron estudiar candidatos
relativistas (calientes) y no relativistas (fríos) de materia oscura
y, comparando las estructuras generadas en la simulación con los
surveys galácticos disponibles, se concluyó que la materia oscura
debía ser no relativista. Tras este éxito inicial, nace el campo de
las simulaciones numéricas de materia oscura\cite{aHistory}.

Actualmente las simulaciones de materia oscura la modelan 
como un fluido \emph{no colisional} que interact\'ua solo gravitacionalmente. 
Esta aproximaci\'on ha sido exitosa explicando el universo a larga
escala, sin embargo se observan inconsistencias para el universo a
escalas menores. 
Por ejemplo, observaciones precisas de galaxias enanas muestran distribuciones
de materia oscura menos densas de lo que predice la aproximaci\'on no
colisional \cite{beyondColl}.
Estos indicios motivan la consideraci\'on de la materia oscura
colisional. 
Por otro lado,  diferentes modelos de materia oscura en física de
partículas tiene límites para la sección transversal de la partícula
de materia oscura lo que implica que a nivel fundamental la materia
oscura debe ser colisional.

El objetivo principal de esta monografía es dejar atr\'as la aproximaci\'on
no colisional para simular materia oscura \emph{colisional} en tres
dimensiones espaciales.
Para esto utilizaremos el m\'etodo de Lattice-Boltzmann 
que  discretiza el espacio de fase de la materia oscura y se
resuelve numéricamente la ecuación de Boltzmann. 


Aunque existen numerosos métodos para simular el espacio de fase como
los métodos de N-cuerpos con Particle Mesh o los esquemas de
integración directa con Volumenes Finitos, elegimos el m\'etodo de
Lattice-Boltzmann porque permite introducir de manera expl\'icita el
t\'ermino colisional.
Otras ventajas del m\'etodo es que es lagrangiano,
conservativo y completamente reversible\cite{integerLatticeDynamics}.
La reversibilidad del algoritmo permite reducir el
costo de memoria a cambio de aumentar el costo computacional. 

Esta monografía es una continuaci\'on directa de la monografía de Sebasti\'an
Franco \cite{franco} quien resolvi\'o la ecuaci\'on de Boltzmann en
una dimensi\'on espacial en el caso no colisional.



%Para el operador colisional, se asume que hay equilibrio dinámico, por
%lo tanto, la distribución de equilibrio es una distribución de
%Fermi-Dirac o de Bose-Einstein. Usando la definición estándar de la
%sección transversal de dispersión y la velocidad relativa entre las
%partículas, se obtiene para el término colisional\cite{mariangela}: 
%\begin{equation}
%\dot{n} = \expval{v \sigma} (n_{eq}^2 - n^2)
%\label{colision}
%\end{equation}
%Introducción a la propuesta de Monografía. Debe incluir un breve resumen del estado del arte del problema a tratar. También deben aparecer citadas todas las referencias de la bibliografía (a menos de que se citen más adelante, en los objetivos o metodología, por ejemplo)


\section{Objetivo General}


Simular el espacio de fase de un fluido de materia oscura colisional con un método de Lattice-Boltzmann.
%Objetivo general del trabajo. Empieza con un verbo en infinitivo.



\section{Objetivos Específicos}

%Objetivos específicos del trabajo. Empiezan con un verbo en infinitivo.

\begin{itemize}
	\item Implementar una simulación de lattice-Boltzmann en 2D con término colisional.
	\item Implementar una simulación de lattice-Boltzmann en 3D con término colisional
	\item Estudiar el comportamiento dinámico de la materia oscura
          con diferentes aproximaciones para el t\'ermino colisional. 
	\item Comparar la evolución del espacio de fase para el fluido
          colisional con su versión no colisional. 
\end{itemize}

\section{Metodología}

%Exponer DETALLADAMENTE la metodología que se usará en la Monografía. 
Partiendo de la naturaleza computacional de la monografía, esta se realizará en un computador de escritorio comercial. No se requiere el uso de un cluster ni de recursos computacionales especiales.


La implementación comienza discretizando el espacio de fase y
definiendo los límites a usar. El espacio de fase se convierte en un
arreglo $2d$ dimensional, donde $d$ es el número de dimensiones
espaciales a simular. 

Tras la discretización del espacio se procede a inicializar la
distribución.
Para esto, cada punto del arreglo $(i,j)$ equivale a una
velocidad, una posición y una densidad de masa. Acto seguido, se
integra respecto a la velocidad para obtener la densidad espacial de
masa. 
 
Con la distribución de masa, se resuelve la ecuación de Poisson a
través del método de transformada de Fourier para calcular el
potencial gravitacional, esto se hace con ayuda de la librería FFTW3
(Fastest Fourier Transform in the West) debido a    su fácil uso y
alta velocidad.

Una vez se tiene el potencial, se deriva numéricamente para calcular
la aceleración y luego se procede a actualizar el espacio de
fase. Primero, se calcula el cambio de velocidad en un tiempo $\dd t$,
luego, usando el operador ''entero más cercano''
$\left\lfloor{.}\right\rceil$, se calcula el traslado en el arreglo
del espacio de fase. Por último, se repite el proceso para el cambio
de posición. 

Adicionalmente, cuando la posición de una partícula sale del arreglo,
se considera que una partícula idéntica entra al arreglo con la misma
velocidad por el extremo opuesto. Cuando la velocidad de una partícula
sale del arreglo se considera que la partícula se perdió. 
%Monografía teórica o computacional: ¿Cómo se harán los cálculos teóricos? ¿Cómo se harán las simulaciones? ¿Qué requerimientos computacionales se necesitan? ¿Qué espacios físicos o virtuales se van a utilizar?



Luego se procede al cálculo del efecto del término colisional sobre la
densidad de n\'umero, $n$, con el siguiente ansatz:

\begin{equation}
\dv{n}{t} = \expval{v \sigma} (n_{eq}^2 - n^2)
\end{equation}

Es importante resaltar que en esta simulación no se va a tener en cuenta
la expansión del universo, esto permite ignorar el término $3 H n$ de la 
expresión clásica para materia oscura térmica. Adicionalmente, se simulará
sistemas en donde la expansión del universo sea insignificante, como la Vía Láctea.

Utilizando distribuciones de Fermi-Dirac y de Bose-Einstein como
distribución de equilibrio, se resuelve para $n_{t+\dd t}$ 
\begin{equation}
n_{t+\dd t} = n_t + \expval{v \sigma} (n_{eq}^2 - n_t^2) \dd t
\end{equation}

Empezaremos primero implementando esta l\'ogica para dos dimensiones
espaciales para luego pasar a tres dimensiones espaciales.

\section{Consideraciones Éticas}

%Esta sección debe incluir los detalles relacionados con aspectos éticos involucrados en el proyecto. Por ejemplo, se puede describir el protocolo establecido para el manejo de datos de manera que se asegure que no habrá manipulación de la información, ni habrá plagio de los mismos. También se puede tener en cuenta si hay algún conflicto de intereses involucrado en el desarrollo del proyecto o se puede detallar si el trabajo está relacionado con las actividades y poblaciones humanas mencionadas en el siguiente link https://ciencias.uniandes.edu.co/investigacion/comite-de-etica. Es importante tener en cuenta que esta sección debe incluir una frase explícita sobre si el proyecto debe pasar o no a estudio del comité de ética de la Facultad de Ciencias.
Dada la naturaleza computacional de la monografía, el proyecto no debe
pasar a estudio por el comité de ética de la Facultad de Ciencias.  
Guardaremos el comportamiento adecuado al citar adecuadamente trabajo
de otras personas. Adicionalmente, todo el software utilizado para la realización de la monografía es de código abierto y está disponible bajo licencia GNU GPL.

\section{Cronograma}

\begin{table}[htb]
	\begin{tabular}{|c|cccccccccccccccc| }
	\hline
	Tareas $\backslash$ Semanas & 1 & 2 & 3 & 4 & 5 & 6 & 7 & 8 & 9 & 10 & 11 & 12 & 13 & 14 & 15 & 16  \\
	\hline
	1 & X & X & X  &   &   &   &   &  &  &   &   &   &   &   &   &   \\
	2 &   &  & X & X & X &  &  &   &   &  &  &  &   &  &  &   \\
	3 &   &   &   &  & X  & X  & X  & X &   &   &   &  &   &   &  &   \\
	4 &  &  &  &  &  &  &  & X & X & X & X &   &   &   &   &   \\
    5 &  &  &  &  &  &  & X & X &  &  &  &   &   &   &   &   \\
	6 &   &   &   &   &  &   &  X & X  &  &   &  X & X &  X & X  & X &   \\
	7 &   &   &   &   &  &   &  &  &  &   &  & & &  & X & X  \\
	\hline
	\end{tabular}
\end{table}
\vspace{1mm}

\begin{itemize}
	\item Tarea 1: implementar la simulación en 2D sin término colisional.
	\item Tarea 2: implementar el término colisional en 2D.
	\item Tarea 3: implementar la simulación en 3D sin término colisional.
    \item Tarea 4: implementar el término colisional en 3D.
    \item Tarea 5: preparar y presentar el avance del 30\%.
    \item Tarea 6: escribir el documento de monografía.
    \item Tarea 7: revisión bibliográfica.
\end{itemize}

\section{Personas Conocedoras del Tema}

%Nombres de por lo menos 3 profesores que conozcan del tema. Uno de ellos debe ser profesor de planta de la Universidad de los Andes.

\begin{itemize}
	\item Jaime Ernesto Forero Romero (Universidad de los Andes)
	\item Carlos Andrés Flórez Bustos (Universidad de los Andes)
	\item Juan Carlos Sanabria Arenas (Universidad de los Andes)
\end{itemize}


\bibliography{bibTes}{}
\bibliographystyle{unsrt}


{\bf Firma del Director} \hfill {\bf Firma del Estudiante}

%\begin{flushright}
%{\bf Firma del Estudiante}
%\end{flushright}
\vspace{1.5cm}




\end{document} 
\grid
\grid
