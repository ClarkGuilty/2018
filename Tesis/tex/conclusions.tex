\chapter{Conclusions}
We simulated the phase space of a collisional dark matter fluid by implementing a Lattice-Boltzmann method that uses a settable relaxation time $\tau$ to model the short range interactions. 

We set the relaxation time equivalent to a \emph{thermally averaged cross-section}  $<\sigma v>$ of $ 3 \e{-26}$ cm$^3$/s, a cosmological matter density of $\Omega_m = 0.312$, and a mass of the dark matter particle of $0.7$ KeV. The origin of these values is discussed in section \ref{metodologiaBGK}. However, we can test dark matter particle candidates by setting different values for $<\sigma v>$ and the mass of the particle.

We successfully implemented a two dimensional phase space simulation in which we tested three initial conditions: Gaussian distribution, Jeans instability, and the collision of two Gaussian halos. 
The Gaussian distribution and the Jeans instability scenario were initially used to test the simulation, and to reproduce previous work done on collisionless dark matter fluids. 
The phase space in the Gaussian initialization evolved into a clockwise rotating spiral, the spatial density kept its Gaussian profile but also started expelling small bumps of mass due to the tails of the velocity distribution.
In the Jeans instability scenario we had essentially the same behavior, but this time instead of a single Gaussian profile we had three (because of how we initialized the Jeans oscillations).
We observed that our implementation is very consistent with previous work and with the expected behavior of the distributions. The Bullet Cluster-like scenario showed that there was no dissipation of energy during the collision, and therefore, the clusters could keep crashing forever in a periodic movement. This was the expected behavior in the absence of a collisional term.

We studied the effects of the collisional term using the Gaussian initial conditions, and Bullet Cluster-like initial conditions, which were simply two Gaussian distributions separated a by given distance. From the Gaussian distribution we concluded that the collisional term reduces the height of the central peak of the distribution, along with the velocities on the peak, it also increases the density in the tails of the spatial distribution.
From the Bullet Cluster initial conditions we observed that after a collision the new greatest distance of the halos is smaller, in other words, there is dissipation of energy due the crash of the halos. After several collision we observed the halos merging, a behavior impossible in a properly implemented no collisional simulation.

We extended the two dimensional simulation to a four dimensional phase space. To test it, we used Gaussian conditions and obtained a complete analog of the two dimensional phase space. We also compared the collisional case with the collisionless one, only to find again that the height of the central peak of the distribution is lower in the collisional case, that there is a higher concentration of mass in the tails of the spatial distribution in the collisional case, and that the velocities in the central peak are lower than in the collisionless scenario.

In the four dimensional simulation we could also observe some problems in due to the low resolution. The arms of the clockwise spiral that phase space forms are not longer resolvable in the four dimensional simulation after a long while. This is an effect of the lattice noise: due to the low concentrations of mass in the arms of the spirals, the lattice noise is enough to blur them into a simple cloud. 

Finally, we extended the four dimensional simulation into a six dimensional one. The increase in the number of dimensions once again brought a decrease in the resolution. During the development of this work we had access to the High Performance Computing Cluster of \emph{Universidad de los Andes}, however, the simulation has such a high memory constrain, that the resolution was low enough for the lattice noise to have a considerable effect in the simulation. Further work most be made in order to reduce the memory requirements. 

We ran Gaussian conditions in the six dimensional simulation, and the general behavior was the same as in the two and four dimensional simulations. However, due to the low resolution we cannot state that the simulation is indeed recovering the continuum Boltzmann equation. 

Summarizing, we simulated the phase space of a collisional dark matter fluid by implementing several Lattice-Boltzmann simulations with an increasing number of dimensions. The two and the four dimensional simulations were succesfully runned with different scenarios. The six dimensional simulation had very high memory constrains that were beyond the available resources during the development of this work.
