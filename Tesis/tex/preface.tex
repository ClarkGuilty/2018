%\author{Javier Alejandro Acevedo Barroso}
%\date{July 2013}
%\title{ Thesis submitted for the degree of Physicist}
%\publishers{Universidad de los Andes}

\begin{titlepage}
   \begin{center}
      \Large\textbf{SIMULATING COLLISIONAL DARK}\\
      \vspace{-4mm}\Large\textbf{MATTER USING A LATTICE}\\
      \vspace{-4mm}\Large\textbf{BOLTZMANN METHOD}\\
\vspace{15mm}
      \Large\textbf{Javier Alejandro}\\
      \vspace{-5mm}\Large\textbf{Acevedo Barroso}\\
\vspace{11mm}      
      \Large\textbf{Advisor:}\\
      \vspace{-5mm}\Large\textbf{Dr. Jaime E. Forero-Romero}\\
\vspace{15mm}      
     \Large\text{A monograph presented for the degree of Physicist.}\\
\vspace{8mm}           
      {\includegraphics[scale=0.11]{imag/logoUniandes.png}}\\
\vspace{3mm}           
      \Large\text{Departamento de Física}\\
      \vspace{-4mm}\Large\text{Facultad de Ciencias}\\
      \vspace{-4mm}\Large\text{Universidad de los Andes}\\
      \vspace{-4mm}\Large\text{Bogotá, Colombia}\\
      \vspace{-1mm}\Large\text{November, 2018}
   \end{center}
\end{titlepage}

\thispagestyle{plain}
\begin{dedication}
\Huge\textit{Dedicado a}\\
%\vspace{-1mm}\large\text{Satán}\\
\vspace{-1mm}\large\text{Mi mamá, pues sin su paciencia y consejo nunca lo habría logrado.}\\
\vspace{-1mm}\large\text{Mi papá, quién me enseñó el valor del arte, la ciencia y la amistad.}\\
\vspace{-1mm}\large\text{Susana, por hacer inolvidables estos cuatro años.}\\
\end{dedication}


\newpage
\thispagestyle{plain}
\begin{center}
      \Large\textbf{Simulating Collisional Dark Matter}\\
      \vspace{-4mm}\Large\textbf{Using a Lattice Boltzmann Method}\\
      %\vspace{-4mm}\Large\textbf{BOLTZMANN METHOD}\\
\vspace{07mm}
      \Large\textbf{Javier Alejandro Acevedo Barroso}\\
      \vspace{-4mm}\Large{November, 2018}\\
      
\vspace{07mm}
      \Large\textbf{Abstract}\\
      \vspace{05mm}\small\justify{
Usually, dark matter is simulated with N-body schemes, that sample the phase space in order to solve the Poisson-Vlasov equation \cite{2012PDU150K}.
These kind of simulations have been essential for the development of modern cosmology and the characterization of dark matter halos. With the development of particle physics, we ultimately expect dark matter to be a particle outside of the standard model of physics. Recent measurements on the aftermath of Galaxy Cluster collisions allow us to constrain the value of the thermally averaged cross section $\crosssection$, motivating the development of \emph{collisional} simulations.\\
On the other hand, Lattice-Boltzmann simulations have been widely used to recreate increasingly complex fluids and boundary conditions, nonetheless, the usual Lattice-Boltzmann scheme does not simulate the entirety of the velocity space, but simply a small number of adventive velocities. \\
Inspired by the work of Philip Mocz, Sauro Succi \cite{integerLatticeDynamics}, and Sebastian Franco \cite{franco}, in which a Lattice-Boltzmann scheme is used to simulate the phase space of a \emph{collisionless} one dimensional dark matter fluid. We implement a Lattice-Boltzmann simulations of the phase space of a \emph{collisional} one, two, and three dimensional dark matter fluid. For the collisional step, we use the BGK approximation modeled by a relaxation time $\tau$ chosen accordingly to recent measurements of the $\crosssection$.}\\
%      \vspace{-4mm}\Large\textbf{Using a Lattice Boltzmann Method}\\	
\end{center}

\newpage
\thispagestyle{plain}
\begin{center}
      \Large\textbf{Simulando Materia Oscura Colisional}\\
      \vspace{-4mm}\large\textbf{Utilizando un Método de Lattice Boltzmann}\\
      %\vspace{-4mm}\Large\textbf{BOLTZMANN METHOD}\\
\vspace{08mm}
      \Large\textbf{Javier Alejandro Acevedo Barroso}\\
      \vspace{05mm}\large{Presentada por el título de Físico.}\\
      \vspace{-4mm}\large{Noviembre, 2018}\\
      
\vspace{09mm}
      \large\textbf{Abstract}\\
      \vspace{05mm}\normalsize{
      Se implementó un método de Lattice Boltzmann para simular el espacio de fase de un fluido de materia oscura \emph{colisional}. Se comparó el espacio de fase del caso no colisional con el caso colisional usando estimados recientes de la \tqt{thermally averaged cross section}}\\
      %      \vspace{-4mm}\Large\textbf{Using a Lattice Boltzmann Method}\\	
\end{center}
\newpage
\thispagestyle{plain}
%      \vspace*{15mm}\Huge\textbf{Agradecimientos}\\
\chapter*{Agradecimientos}
      \normalsize{Utilizando un Método de Lattice Boltzmann}\\
