\chapter{Introduction}
In this work, we simulate the phase space of a collisional dark matter fluid, for that, it is essential to know some concepts and computational techniques. In this chapter, we present all the necessary knowledge for the understanding (and development) of this work.
\section{Dark Matter, or the Missing Mass Problem}
Modern cosmology describes the universe as being composed of two fundamental types of energy: dark energy and matter\footnote{In relativity, mass and energy are equivalent.}, with dark energy being associate with a cosmological constant and matter being divided into two categories: dark matter and standard model matter\footnote{Which is very often called \tqt{Baryonic matter} due to Baryions being the largest fraction of this mass.}. The energy density in the universe is $69\%$ dark energy and $31\%$ matter.

Standard model matter is all the particles whose interactions can be properly described by the standard model, that includes: Protons, Electrons, Atoms and naturally, any structure they form, like Humans or Stars.
On the other hand, dark matter is all the matter we measure from astrophysical sources which cannot be explained by baryonic matter. We know of the existence of dark matter entirely from astrophysical evidence, during this section we are going to do an historical review of such evidence.
\section{Types of Dark Matter}
\section{The Boltzmann Equation}
\section{Lattice Automata and Lattice Boltzmann}
\section{BGK Approximation}
